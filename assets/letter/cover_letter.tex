% \pdfminorversion=4
\documentclass[a4paper,11pt,twoside]{article}
\usepackage[top=8mm,bottom=12mm,left=20mm,right=20mm, includehead, includefoot]{geometry}
\usepackage{pdflscape}
%\usepackage{lscape}
\usepackage[T1]{fontenc}
\usepackage[utf8]{inputenc}

\usepackage{times}
%\usepackage{helvet}
%\renewcommand{\familydefault}{\sfdefault}

\usepackage[yyyymmdd]{datetime}

\usepackage{hyperref}
\usepackage{amsmath,amsfonts,amssymb,amscd,amsthm}

\usepackage[makeroom]{cancel}
%\usepackage[latin1]{inputenc}
\usepackage[utf8]{inputenc}
\usepackage[numbers]{natbib}
\usepackage{color}
\usepackage[usenames,dvipsnames]{xcolor}
\usepackage{doi}
\usepackage{graphicx}
\usepackage{booktabs}
%\usepackage{glossaries}
%\usepackage{times}
\usepackage{multicol}
\usepackage{longtable}
\usepackage{booktabs}
\usepackage[labelfont={bf,small},font=small]{caption}
%
\usepackage[vlined]{algorithm2e}
%\SetKw{KwRet}{return}
\SetKwProg{Fn}{Procedure}{}{end}
%
%\RestyleAlgo{boxed}
\SetAlCapSkip{0.5em}
\newcommand{\SetSmall}[1]{{\small #1 }}
%\newcommand{\AlCapFnt}{small}
\SetAlgoCaptionLayout{SetSmall}

\usepackage{lastpage}

\usepackage{sectsty}
\sectionfont{\normalsize\selectfont}
\subsectionfont{\normalsize\selectfont}
\subsubsectionfont{\normalsize\selectfont}
\paragraphfont{\em\normalsize\selectfont}
\usepackage[compact]{titlesec}

\usepackage{enumitem}

%\usepackage{tikz}

\newcommand{\mytitle}{Cover letter}

\hypersetup{
    bookmarks=true,
    breaklinks=true,
    bookmarksopen=true,
    pdftitle={\mytitle},    % title
    pdfauthor={Santiago Badia}, % author
    colorlinks=true,        % false: boxed links; true: colored links
    linkcolor=black,        % color of internal links
    citecolor=black,        % color of links to bibliography
    filecolor=black,        % color of file links
    urlcolor=black          % color of external links
}

\setlength{\parindent}{0pt}

\usepackage{autonum}

\usepackage{fancyhdr}
\fancyhead{}
\fancyfoot{}
\fancyfoot[C]{\color{gray} Page \thepage ~of {\color{gray}\pageref*{sec:endpage}}}
\renewcommand{\headrulewidth}{0pt}
\renewcommand{\footrulewidth}{0pt}

\setlength{\tabcolsep}{0pt}
\newlength{\colA}
\newlength{\colB}
\newlength{\colC}
\newlength{\colS}

\newcommand{\todo}{\color{red}}

\begin{document}
\pagestyle{fancy}

\begin{flushleft}
    Santiago Badia -- \texttt{santiago.badia@monash.edu} \\
    School of Mathematics -- Monash University \\
    Address: 9 Rainforest Walk, Clayton, Victoria 3800 (Australia) \\
\end{flushleft}

\begin{flushright}
    \includegraphics[height=2.0cm]{monash-logo.png}
\end{flushright}

\begin{flushright}
    Melbourne, 15th June 2021
\end{flushright}

\begin{flushleft}
To the editorial board of the journal\\ Computer Methods in Applied Mechanics
and Engineering
\end{flushleft}

Dear Editor:\\

We hereby submit the manuscript entitled \emph{"Linking ghost penalty and aggregated unfitted methods"} to be
considered for publication in Computer Methods in Applied Mechanics and
Engineering. This is an original contribution that has not been published before
and that is not currently under review at any other journal.\\

In this paper, we explore, for the first time, the links between 
the ghost penalty and aggregated unfitted finite element method (AgFEM). 
Out of this analysis, we design a new set of unfitted finite element schemes
that are ghost penalty methods (rely on a stabilisation term) but this penalty
is defined in terms of a discrete extension operator, as in AgFEM.
The resulting method converges to AgFEM as the penalty parameter increases and
is free of a locking phenomenon of previous ghost penalty methods that is observed and explained in this work.\\

We also perform for the first time a thorough comparison of ghost penalty and AgFEM methods for 
different geometries, orders, dimensions, problems and boundary conditions. 
We analyse condition number bounds, convergence rates and sensitivity of ghost penalty methods with respect
to numerical parameters. The results obtained with the new methods are remarkable.
They are free of the spurious locking observed in previous schemes and their accuracy is highly insensitive to numerical parameters.\\

We expect this paper to provide better understanding and useful
guidance for the application, design, and choice of unfitted finite element methods.\\

Thank you for your valuable time and consideration. \\

Yours faithfully,
\vspace{0.6cm}
% Optional
% \begin{flushleft}
% \includegraphics[scale=0.3]{signature.pdf}
% \end{flushleft}
\vskip0.6cm
Santiago Badia

\label{sec:endpage}
\end{document}
